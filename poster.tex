\documentclass{tikzposter}
\tikzposterlatexaffectionproofoff

\usepackage{hyperref}
\usepackage{doi}
\usepackage{qrcode}

\makeatletter
\renewcommand\TP@maketitle{%
   \begin{minipage}{0.8\linewidth}
        \color{titlefgcolor}
        {\bfseries \Huge \sc \@title \par}
        \vspace*{1em}
        {\huge \@author \par}
        \vspace*{1em}
        {\LARGE \@institute}
    \end{minipage}
    \hfill
    \begin{minipage}{0.2\linewidth}
       \centering
       \begin{tikzpicture}
           \begin{scope}[scale=0.09, y={(0,-1)}]
               \input{uoa.tex}
           \end{scope}
           \begin{scope}[scale=0.07, y={(0,-1)}, shift={(7,65)}]
               \input{awc.tex}
           \end{scope}
       \end{tikzpicture}
    \end{minipage}
}
\makeatother

\title{Sifting Through Tangled Trees: Bayesian Reconstruction of Cophylogenies}
\author{\underline{Arman Bilge}, Timothy Vaughan, and Alexei J. Drummond}
\institute{\Large email: \href{mailto:abil933@aucklanduni.ac.nz}{\texttt{abil933@aucklanduni.ac.nz}}}

\defineblockstyle{MySlide}{
    titlewidthscale=1, bodywidthscale=1, titleleft,
    titleoffsetx=0pt, titleoffsety=0pt, bodyoffsetx=0pt, bodyoffsety=0pt,
    bodyverticalshift=0pt, roundedcorners=0, linewidth=0pt, titleinnersep=1cm,
    bodyinnersep=1cm
}{
    \ifBlockHasTitle%
        \draw[draw=none, left color=blocktitlebgcolor, right color=blocktitlebgcolor]
           (blocktitle.south west) rectangle (blocktitle.north east);
    \fi%
    \draw[draw=none, fill=blockbodybgcolor] %
        (blockbody.north west) [rounded corners=30] -- (blockbody.south west) --
        (blockbody.south east) [rounded corners=0]-- (blockbody.north east) -- cycle;
}

\usetheme{Autumn}
\usecolorstyle{Russia}
\definecolor{uoanavy}{RGB}{0,60,127}
\colorlet{titlebgcolor}{uoanavy}
\colorlet{blocktitlebgcolor}{uoanavy}
\useblockstyle{MySlide}
\renewcommand{\familydefault}{\sfdefault}
\usepackage{sfmath}

\usepackage{multicol}
\usepackage{enumitem}
\setlist[itemize]{label=$\triangleright$, labelsep=12pt}

% Plots
\usetikzlibrary{arrows.meta}
\usepackage{pgfplots}
\usepackage{pgfplotstable}
\pgfplotsset{width=32cm,
             compat=newest,
             jitter/.style={
                 x filter/.code={\pgfmathparse{\pgfmathresult+rnd*#1-#1/2}}
             },
             jitter/.default=0,
             select coords between index/.style 2 args={
                 x filter/.code={
                     \ifnum\coordindex<#1\def\pgfmathresult{}\fi
                     \ifnum\coordindex>#2\def\pgfmathresult{}\fi
                 }
             }
 }

\makeatletter
\pgfarrowsdeclare{center*}{center*}
{
  \pgfarrowsleftextend{+-.5\pgflinewidth}
  \pgfutil@tempdima=0.4pt%
  \advance\pgfutil@tempdima by.2\pgflinewidth%
  \pgfarrowsrightextend{4.5\pgfutil@tempdima}
}
{
  \pgfutil@tempdima=0.4pt%
  \advance\pgfutil@tempdima by.2\pgflinewidth%
  \pgfsetdash{}{+0pt}
  \pgfpathcircle{\pgfqpoint{4.5\pgfutil@tempdima}{0bp}}{4.5\pgfutil@tempdima}
  \pgfusepathqfillstroke
}
\pgfarrowsdeclare{centero}{centero}
{
  \pgfarrowsleftextend{+-.5\pgflinewidth}
  \pgfutil@tempdima=0.4pt%
  \advance\pgfutil@tempdima by.2\pgflinewidth%
  \pgfarrowsrightextend{4.5\pgfutil@tempdima}
}
{
  \pgfutil@tempdima=0.4pt%
  \advance\pgfutil@tempdima by.2\pgflinewidth%
  \pgfsetdash{}{+0pt}
  \pgfpathcircle{\pgfqpoint{4.5\pgfutil@tempdima}{0bp}}{4.5\pgfutil@tempdima}
  \pgfusepathqstroke
}
\makeatother

\usepackage{mathtools}
\newcommand{\dd}{\, \mathrm{d}}
\renewcommand{\vec}[1]{\ensuremath{\boldsymbol{\mathbf{#1}}}}
\newcommand{\mat}[1]{\ensuremath{\boldsymbol{\mathbf{#1}}}}
\newcommand{\op}[1]{\ensuremath{\boldsymbol{\mathbf{#1}}}}
\newcommand{\norm}[1]{\ensuremath{\mathcal{N}\left(#1\right)}}

\usepackage{algorithm}
\usepackage{algpseudocode}
\algrenewcomment[2][.41\linewidth]{\leavevmode\hfill\makebox[#1][l]{\(\triangleright\)~#2}}

\usepackage[none]{hyphenat}
\frenchspacing

\begin{document}

    \maketitle

    \begin{columns}

        \column{0.5}

        \block{Motivation}{
            \flushleft
            \begin{itemize}
                \item Horizontal gene transfer etc.
            \end{itemize}
        }

        \block{The Cophylogeny Model}{
            \centering
            \begin{tikzpicture}
                \draw[blue] (0,0) -- (10,10);
                \draw[blue] (4,0) -- (2,2);
                \draw[blue] (8,0) -- (4,4);
                \draw[blue] (12,0) -- (14,2);
                \draw[blue] (16,0) -- (8,8);
                \draw[red] (1,0) -- (11,10);
                \draw[white,-center*] (9,8) -- (9,8);
                \draw[red,-centero] (17,0) -- (9,8);
                \draw[white,-center*] (15,2) -- (15,2);
                \draw[red,-centero] (13,0) -- (15,2);
                \draw[red] (7,3) -- (14,3);
                \draw[red,arrows={Latex[scale=1]-}] (10,3) -- (14,3);
                \draw[red] (10,0) -- (7,3);
                \draw[white,-center*] (5,4) -- (5,4);
                \draw[red,-centero] (9,0) -- (5,4);
                \draw[white,arrows=-center*] (3,2) -- (3,2);
                \draw[red,arrows=-centero] (4,1) -- (3,2);
                \node[red, rotate=135] at (4,1) {\Huge\textsf{\textbf{x}}};
                \draw[red] (18,0) -- (17,1);
                \draw[red,center*-] (16,1) -- (17,1);
            \end{tikzpicture}

            \innerblock{Cospeciation}{A host and all of its guests speciate simultaneously}
            \innerblock{Duplication}{A guest speciates and remains on the same host at rate $\lambda$}
            \innerblock{Host-Switch}{A guest speciates and selects a new host at rate $\tau$}
            \innerblock{Loss}{A guest dies at rate $\mu$}
            \innerblock{Sampling}{A guest is sampled with probability $\rho$}
        }

        \block{The Particle Markov chain Monte Carlo Algorithm}{

            \begin{enumerate}

                \item Approximate the probability of the guest tree and reconciliation
                \begin{enumerate}
                    \item Create $N$ particles, each with an population of guests
                \end{enumerate}
            \end{enumerate}

        }

        \column{0.5}

        \block{Future Work}{
            \begin{itemize}
                \item Relax deterministic host tree
            \end{itemize}
        }

        \block{Acknowledgements}{
            \flushleft
            \begin{itemize}
                \item Members of the Computational Evolution Groups at The University of Auckland and ETH~Z\"urich, especially David Rasmussen, Gabriel Leventhal, and Tanja Stadler
                \item Allan Wilson Centre for travel funding
            \end{itemize}
        }

        \block{References}{
            \flushleft

        }

        \block{}{
            \begin{multicols*}{2}

                \flushleft

                \qrcode[height=5cm]{http://dx.doi.org/10.5281/zenodo.22305}
                Download this poster.

                \vspace{25pt} \texttt{\doi{10.5281/zenodo.22305}}

                \columnbreak

                \qrcode[height=5cm]{http://git.io/vOLj1}
                Fork the source code.

                \vspace{25pt} \url{http://git.io/vOLj1}

            \end{multicols*}
        }

    \end{columns}

\end{document}